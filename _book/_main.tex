% From https://github.com/UWIT-IAM/UWThesis
\documentclass[print]{nuthesis}
\usepackage{amssymb, amsthm, amsmath, amsfonts}
\usepackage{wasysym}
\usepackage{mathrsfs}
% \usepackage{hyperref}
\usepackage{graphicx}
\usepackage{lineno}
\usepackage[colorinlistoftodos]{todonotes}
\usepackage{listings}
%\usepackage{breqn}
\usepackage{cancel, enumerate}
\usepackage{rotating, environ}
\usepackage{caption}
\usepackage{subcaption}
\usepackage[inline]{enumitem}
\usepackage{dirtree}

\newtheorem{thm}{Theorem}
\newtheorem{defn}{Definition}
\newtheorem{prop}{Proposition}
\newtheorem{lemma}{Lemma}
\newtheorem{cor}{Corollary}

% Syntax highlighting #22

%% https://github.com/rstudio/rmarkdown/issues/1649
\newlength{\cslhangindent}
\setlength{\cslhangindent}{1.5em}
\newenvironment{CSLReferences}%
{\setlength{\parindent}{0pt}%
\everypar{\setlength{\hangindent}{\cslhangindent}}\ignorespaces}%
{\par}

% fix for pandoc 1.14
\providecommand{\tightlist}{%
  \setlength{\itemsep}{0pt}\setlength{\parskip}{0pt}}

%% something about tables, from https://github.com/ismayc/thesisdown/issues/122
\usepackage{calc}

%% for copyright symbol
\usepackage{textcomp}

%% to allow to rotate pages to landscape
\usepackage{lscape}
%% to adjust table column width
\usepackage{tabularx}

% suppress bottom page numbers on first page of each chapter
% because they overlap with text
\usepackage{etoolbox}
\patchcmd{\chapter}{plain}{empty}{}{}

%% for more attractive tables
\usepackage{booktabs}
\usepackage{longtable}


\usepackage{graphicx}


% Double spacing, if you want it.
\def\dsp{\def\baselinestretch{2.0}\large\normalsize}
% \dsp

% If the Grad. Division insists that the first paragraph of a section
% be indented (like the others), then include this line:
\usepackage{indentfirst}

%%%%%%%%%%%%%%%%%%
% If you want to use "sections" to partition your thesis
% un-comment the following:
%
\counterwithout{section}{chapter}
\setsecnumdepth{subsubsection}
\def\sectionmark#1{\markboth{#1}{#1}}
\def\subsectionmark#1{\markboth{#1}{#1}}
\renewcommand{\thesection}{\arabic{section}}
\renewcommand{\thesubsection}{\thesection.\arabic{subsection}}
\makeatletter
\let\l@subsection\l@section
\let\l@section\l@chapter
\makeatother

\renewcommand{\thetable}{\arabic{table}}
\renewcommand{\thefigure}{\arabic{figure}}

%%%%%%%%%%%%%%%%%%


%% Stuff from https://github.com/suchow/Dissertate

% The following line would print the thesis in a postscript font

% \usepackage{natbib}
% \def\bibpreamble{\protect\addcontentsline{toc}{chapter}{Bibliography}}

\setcounter{tocdepth}{1} % Print the chapter and sections to the toc
% controls depth of table of contents (toc): 0 = chapter, 1 = section, 2 = subsection

\usepackage{natbib}


% commands and environments needed by pandoc snippets
% extracted from the output of `pandoc -s`
%% Make R markdown code chunks work
\usepackage{array}
\usepackage{amssymb,amsmath}
\usepackage{ifxetex,ifluatex}
\ifxetex
  \usepackage{fontspec,xltxtra,xunicode}
  \defaultfontfeatures{Mapping=tex-text,Scale=MatchLowercase}
\else
  \ifluatex
    \usepackage{fontspec}
    \defaultfontfeatures{Mapping=tex-text,Scale=MatchLowercase}
  \else
    \usepackage[utf8]{inputenc}
  \fi
\fi
\usepackage{color}
\usepackage{fancyvrb}


\ifxetex
  \usepackage[setpagesize=false, % page size defined by xetex
              unicode=false, % unicode breaks when used with xetex
              xetex,
              colorlinks=true,
              linkcolor=blue]{hyperref}
\else
  \usepackage[unicode=true,
              colorlinks=true,
              linkcolor=blue]{hyperref}
\fi
\hypersetup{breaklinks=true, pdfborder={0 0 0}}
\setlength{\parindent}{20pt}
\setlength{\parskip}{6pt plus 2pt minus 1pt}
\setlength{\emergencystretch}{3em}  % prevent overfull lines
\setcounter{secnumdepth}{2} %% controls section numbering, e.g. 1 or 1.2, or 1.2.3



%  ----  Text Colors  ------------------------------------------
%
% Assign colors to writers for review
%
\newcommand{\ear}[1]{{\textcolor{blue}{#1}}}
\newcommand{\svp}[1]{{\textcolor{RedOrange}{#1}}}
\newcommand{\rh}[1]{{\textcolor{Green}{#1}}}
%
\raggedright
\setlength{\parindent}{0.5cm}
\usepackage[capitalise]{cleveref}
\newcommand\pcref[1]{(\cref{#1})}

%  --- Code chunk font size -----------------------------------------------
% https://stackoverflow.com/questions/38323331/code-chunk-font-size-in-beamer-with-knitr-and-latexhttps://stackoverflow.com/questions/38323331/code-chunk-font-size-in-beamer-with-knitr-and-latex

%% change fontsize of R code
% \let\oldShaded\Shaded
% \let\endoldShaded\endShaded
% \renewenvironment{Shaded}{\footnotesize\oldShaded}{\endoldShaded}
% 
% %% change fontsize of output
% \let\oldverbatim\verbatim
% \let\endoldverbatim\endverbatim
% \renewenvironment{verbatim}{\footnotesize\oldverbatim}{\endoldverbatim}


\begin{document}
% \linenumbers{}
%% Start formatting the first few special pages
%% frontmatter is needed to set the page numbering correctly
\frontmatter
%% from thesisdown
% To pass between YAML and LaTeX the dollar signs are added by CII
\title{HUMAN PERCEPTION OF EXPONENTIALLY INCREASING DATA DISPLAYED ON A LOG SCALE}
\author{Emily Anna Robinson}
\adviser{Susan VanderPlas and Reka Howard}
\adviserAbstract{}
\major{Statistics}
\degreemonth{August}
\degreeyear{2022}
% \copyrightpage
%%
%% For most people the defaults will be correct, so they are commented
%% out. To manually set these, just uncomment and make the needed
%% changes.
%% \college{Your college}
%% \city{Your City}
%%
%% For most people the following can be changed with a class
%% option. To manually set these, just uncomment the following and
%% make the needed changes.
%% \doctype{Thesis or Dissertation}
%% \degree{Your degree}
%% \degreeabbreviation{Your degree abbr.}
%%
%% Now that we know everything we need, we can generate the title page
%% itself.
%%
\maketitle


\begin{abstract}
    Log scales are often used to display data over several orders of magnitude within one graph. During the COVID-19 pandemic, we have seen both the benefits and the pitfalls of using log scales to display case counts. Three graphical experimental tasks were conducted to evaluate the impact our choice of scale has on human perception of exponentially increasing trends. The first experiment evaluates whether our ability to perceptually notice differences in exponentially increasing trends is impacted by the choice of scale. We conducted a visual inference experiment in which participants were shown a series of lineup plots (consisting of 19 null panels and 1 target panel) and asked to identify the panel that was most different from the others. Our results indicated that when there was a large difference in curvature between the target plot and null plots, the choice of scale had no impact and participants accurately differentiated between the two curves on both the linear and log scale. However, displaying exponentially increasing data on a log scale improved the accuracy of differentiating between models with slight curvature differences. An exception occurred when identifying a plot with curvature embedded in surrounding plots closely relating to a linear trend, indicating that it is easy to identify a curve in a group of lines but much harder to identify a line in a group of curves. The use of visual inference to identify these guidelines suggests that there are \emph{perceptual} advantages to log scales when differences are subtle. Our other experimental tasks focus on determining whether there are cognitive disadvantages to log scales: do log scales make it harder to make use of graphical information? We conducted a graphical task similar to the New York Times ``You Draw It'' page to test an individual's ability to use and make predictions for exponentially increasing data. We asked participants to draw a line using their computer mouse through the increasing exponential trend shown on both scales. In addition to differentiation and prediction of exponentially increasing data, we conduct an experimental task to test an individuals' ability to translate a graph of exponentially increasing data into real value quantities and extend their estimations by making comparisons. The results of our experimental tasks allow us to provide guidelines for readers to actively choose which of many possible graphics to draw, according to some set of design choices, to ensure that our charts are effective. \emph{(399 words; 350 word limit)}
\end{abstract}

%% Optional
%% \begin{copyrightpage}
%% \end{copyrightpage}

%% Optional
% \begin{dedication}
% Dedicated to\ldots{}
% \end{dedication}

%%%%%%%%%%%%%%%%%%%
% Acknowledgments
%%%%%%%%%%%%%%%%%%%
\begin{acknowledgments}
Thank you to all my people!
\end{acknowledgments}
%%%%%%%%%%%%%%%%%%%

%%%%%%%%%%%%%%%%%%%
% Grant Information
%%%%%%%%%%%%%%%%%%%
% \begin{grantinfo}
%     % Add any grant info here
% \end{grantinfo}

%%%%%%%%%%%%%%%%%%%
% ToC
%%%%%%%%%%%%%%%%%%%
\tableofcontents

%%%%%%%%%%%%%%%%%%%
% List of Figures
%%%%%%%%%%%%%%%%%%%
\listoffigures
\listoftables

%%%%%%%%%%%%%%%%%%%
% Start of the document
%%%%%%%%%%%%%%%%%%%
\mainmatter


\hypertarget{literature-review}{%
\chapter{Literature Review}\label{literature-review}}

Editing text colors: \ear{Emily's editing color.} Emily may also use mostly black text as well. \svp{Susan's editing color.} \rh{Reka's editing color.}

\begin{itemize}
\tightlist
\item
  Introduction
\end{itemize}

With the advancement of technology, access to data is more readily available more than ever before.
With this abundance of data comes the responsibility to make sense out of the numbers in order to convey the message clearly.
Graphics are a useful tool for displaying and communicating information (Vanderplas, Cook, \& Hofmann, 2020).
Researchers include graphics to communicate their results in scientific publications and news sources rely on graphics to convey news stories to the public {[}\ear{CITE ME}{]}.

\begin{itemize}
\tightlist
\item
  Motivation (COVID-19)

  \begin{itemize}
  \tightlist
  \item
    Charlotte (2020)
  \item
    Romano, Sotis, Dominioni, \& Guidi (2020)
  \item
    Rost (2020)
  \item
    Bavel et al. (2020)
  \item
    Fagen-Ulmschneider (2020)
  \item
    Financial Times (2020)
  \end{itemize}
\end{itemize}

At the beginning of the SARS-NCOV-2 pandemic (COVID-19), we saw an influx of dashboards being developed to display case counts, transmission rates, and outbreak regions (Charlotte, 2020); mass media routinely showed charts to share information with the public about the progression of the pandemic (Romano, Sotis, Dominioni, \& Guidi, 2020).
People began seeking out graphical displays of COVID-19 data as a direct result of these pieces of work (Rost, 2020); providing increased and ongoing exposure to these graphics over time.
Many of these graphics helped guide decision makers to implement policies such as shut-downs or mandated mask wearing, as well as facilitated communication with the public to increase compliance (Bavel et al., 2020).

\begin{itemize}
\tightlist
\item
  General Sensation and Perception
\item
  General Graphics Research

  \begin{itemize}
  \tightlist
  \item
    Vanderplas, Cook, \& Hofmann (2020)
  \item
    Unwin (2020)
  \end{itemize}
\end{itemize}

With the increasing importance graphics play in our everyday lives, we must actively choose which of many possible graphics to draw, according to some set of design choices, to ensure that our charts are effective, as suggested in (Unwin, 2020).

\begin{itemize}
\tightlist
\item
  Logrithms

  \begin{itemize}
  \tightlist
  \item
    Menge et al. (2018)
  \item
    Munroe (2013)
  \item
    Heckler, Mikula, \& Rosenblatt (2013)
  \item
    Varshney \& Sun (2013)
  \item
    Siegler \& Braithwaite (2017)
  \item
    Dehaene, Izard, Spelke, \& Pica (2008)
  \end{itemize}
\item
  Visual Inference through Lineups

  \begin{itemize}
  \tightlist
  \item
    Best, Smith, \& Stubbs (2007)
  \item
    Susan VanderPlas \& Hofmann (2017)
  \item
    Hofmann, Follett, Majumder, \& Cook (2012)
  \item
    Loy, Follett, \& Hofmann (2016)
  \item
    Wickham, Cook, Hofmann, \& Buja (2010)
  \item
    Majumder, Hofmann, \& Cook (2013)
  \item
    Vanderplas, Röttger, Cook, \& Hofmann (2020)
  \end{itemize}
\item
  Prediction with You Draw It

  \begin{itemize}
  \tightlist
  \item
    William A. Wagenaar \& Sagaria (1975)
  \item
    Mackinnon \& Wearing (1991)
  \item
    Jones (1977)
  \item
    Jones (1979)
  \item
    W. A. Wagenaar \& Timmers (1978)
  \item
    Timmers \& Wagenaar (1977)
  \item
    Mosteller, Siegel, Trapido, \& Youtz (1981)
  \item
    Finney (1951)
  \end{itemize}
\item
  Numerical Translation and Estimation
\end{itemize}

\emph{Text below copied from JSM 2021 Student Paper}

When faced with data which spans several orders of magnitude, we must decide whether to show the data on its original scale (compressing the smaller magnitudes into relatively little area) or to transform the scale and alter the contextual appearance of the data.
One common solution is to use a log scale transformation to display data over several orders of magnitude within one graph.
Logarithms make multiplicative relationships additive, showing elasticities and other proportional changes, and also linearize power laws (Menge et al., 2018).
When presenting log scaled data, it is possible to use either untransformed scale labels (for example, values of 1, 10 and 100 are equally spaced along the axis) or log transformed scale labels (for example, 0, 1, and 2, showing the corresponding powers of 10).
We have recently experienced the benefits and pitfalls of using log scales as COVID-19 dashboards displayed case count data on both the log and linear scale Financial Times (2020).
In spring 2020, during the early stages of the COVID-19 pandemic, there were large magnitude discrepancies in case counts at a given time point between different geographic regions (e.g.~states and provinces as well as countries and continents).
During this time, we saw the usefulness of log scale transformations showing case count curves for areas with few cases and areas with many cases within one chart.
As the pandemic evolved, and the case counts were no longer spreading exponentially, graphs with linear scales seemed more effective at spotting early increases in case counts that signaled more localized outbreaks.
This is only one recent example of a situation in which both log and linear scales are useful for showing different aspects of the same data; there are long histories of using log scales to display results in ecology, psychophysics, engineering, and physics
Heckler, Mikula, \& Rosenblatt (2013).
In fact, research shows that we do not perceive exponential growth accurately (whether information is presented in graphical or tabular form): we systematically under-estimate exponential growth.

Research suggests our perception and mapping of numbers to a number line is logarithmic at first, but transitions to a linear scale later in development, with formal mathematics education Dehaene, Izard, Spelke, \& Pica (2008).
This transition to linear scales occurs first in small numbers (e.g.~1-10) and then gradually expands to higher orders of magnitude; thus, the logarithmic intuition about numbers in children is often more noticeable on scales in the thousands to hundreds of thousands.
If we perceive logarithmically by default, it is a natural (and presumably low-effort) way to display information and should be easy to read and understand/use.
In fact, early studies explored the estimation and prediction of exponential growth, finding that growth is underestimated when presented both numerically and graphically but that numerical estimation is more accurate than graphical estimation for exponential curves (William A. Wagenaar \& Sagaria, 1975).
One way to improve estimation of increasing exponential trends is to provide immediate feedback to participants about the accuracy of their current predictions (Mackinnon \& Wearing, 1991).
While prior contextual knowledge or experience with exponential growth does not improve estimation, instruction on exponential growth reduces the underestimation: participants adjust their initial starting value but not their perception of growth rate Jones (1977).
Estimation was shown to improve when subjects were presented with decreasing exponential functions (Timmers \& Wagenaar, 1977).
Jones (1977),Jones (1979) and W. A. Wagenaar \& Timmers (1978) propose competing polynomial models for the perception and extrapolation of exponential series.
It seems that estimation is a two-stage process: first, we identify the type of curve and direction and then, we use that information for prediction (Best, Smith, \& Stubbs, 2007).

Our inability to accurately predict exponential growth might also be addressed by log transforming the data, however, this transformation introduces new complexities; most readers are not mathematically sophisticated enough to intuitively understand logarithmic math and translate that back into real-world effects.
In Menge et al. (2018), ecologists were surveyed to determine how often ecologists encounter log scaled data and how well ecologists understand log scaled data when they see it in the literature.
Participants were presented two relationships displayed on linear-linear scales, log-log scales with untransformed values, or log--log scales with log transformed values.
Menge et al. (2018) propose three types of misconceptions participants encountered when presented data on log-log scales:
`hand-hold fallacy,' `Zeno's zero fallacy,' and `watch out for curves fallacies.' These misconceptions are a result of linear extrapolation assuming that a line in log-log space represents a line instead of the power law in linear-linear space. The study found that in each of these scenarios, participants were confident in their incorrect responses, indicating incorrect knowledge rather than a lack of knowledge.

The `hand-hold falacy' stems from the misconception that steeper slopes in log-log relationships are steeper slopes in linear-linear space.
In fact, it is not only the slope that matters, but also the intercept and the location on the horizontal axis since a line in log-log space represents a power law in linear-linear space (i.e.~linear extraploation).
Emerging from `Zeno's zero fallacy' is the misconception that positively sloped lines in log-log space can imply a non-zero value of y when x is zero. This is never true as postively sloped lines in log-log space actually imply that y = 0 when x = 0.
This misconception again is a result of linear extrapolation assuming that a line in log-log space represents a line instead of the power law in linear-linear space.
The last misconception, `watch out for curves fallacies' encompases three faults: (1) lines in log-log space are lines in linear-linear space, (2) lines in log-log space curve upward in linear-linear space, and (3) curves in log-log space have the same curvature in linear-linear space.
Linear extrapolation is again responsible for the first and third faults while the second fault is a result of error in thinking that log-log lines represent power laws (which are exponential relationships), and all exponential relationships curve upward; this is only true when the log-log slope is greater than one. Menge et al. (2018) found that in each of these scenarios, participants were confident in their incorrect responses, indicating incorrect knowledge rather than a lack of knowledge.

In order to provide a set of principles to guide design choices, we must evaluate these design choices through the use of graphical tests.
These tests may take many forms: identifying differences in graphs, reading information off of a chart accurately, using data to make correct real-world decisions, or predicting the next few observations.
All of these types of tests require different levels of use and manipulation of the information presented in the chart.

\hypertarget{lineups}{%
\chapter{Perception through lineups}\label{lineups}}

\hypertarget{introduction}{%
\section{Introduction}\label{introduction}}

To lay a foundation for future exploration of the use of log scales, we begin with the most fundamental ability to identify differences in charts; this does not require that participants understand exponential growth, identify log scales, or have any mathematical training.
Instead, we are simply testing the change in perceptual sensitivity resulting from visualization choices. In Best, Smith, \& Stubbs (2007), the authors explored whether discrimination between curve types is possible.
They found that accuracy of identifying the correct curve type is higher when nonlinear trends were presented (e.g.~it's hard to say something is linear, but easy to say that it is not linear) and that accuracy is higher with low additive variability.

A statistic is a numerical function which summarizes the data; by this definition, graphs are visual statistics.
To evaluate a graph, we have to run our statistic through a visual evaluation - a person.
If two different methods of presenting data result in qualitatively different results when evaluated visually, then we can conclude that the visual statistics are significantly different.
Recent graphical experiments have utilized statistical lineups to quantify the perception of graphical design choicesLoy, Follett, \& Hofmann (2016).
Statistical lineups provide an elegant way of combining perception and statistical hypothesis testing using graphical experiments Vanderplas, Röttger, Cook, \& Hofmann (2020).
`Lineups' are named after the `police lineup' of criminal investigations where witnesses are asked to identify the criminal from a set of individuals.
Similarly, a statistical lineup is a plot consisting of smaller panels; the viewer is asked to identify the plot of the real data from among a set of decoy null plots.
A statistical lineup typically consists of 20 panels - 1 target panel and 19 null panels (Figure \ref{fig:lineup-example}).
If the viewer can identify the target panel embedded within the set of null panels, this suggests that the real data is visually distinct from data generated under the null model.
Crowd sourcing websites such as Amazon Mechanical Turk, Reddit, and Prolific allow us to collect responses from multiple viewers.
In this paper, we use statistical lineups to test our ability to differentiate between exponentially increasing curves with differing levels of curvature, using linear and log scales. Vanderplas, Röttger, Cook, \& Hofmann (2020) provides an approach for calculating visual p-values utilizing a `rorschach' lineup which consists solely of null panels.

\begin{figure}

{\centering \includegraphics[width=\linewidth]{_main_files/figure-latex/lineup-example-1} 

}

\caption{Lineup example}\label{fig:lineup-example}
\end{figure}

\hypertarget{data-generation}{%
\section{Data Generation}\label{data-generation}}

In this study, both the target and null data sets were generated by simulating data from an exponential model; the models differ in the parameters selected for the null and target panels.
In order to guarantee the simulated data spans the same domain and range of values, we implemented a domain constraint of \(x\in [0,20]\) and a range constraint of \(y\in [10,100]\) with \(N = 50\) points randomly assigned throughout the domain and mapped to the y-axis using the exponential model with the selected parameters.
These constraints provide some assurance that participants who select the target plot are doing so because of their visual perception differentiating between curvature or growth rate rather than different starting or ending values.

We simulated data based on a three-parameter exponential model with multiplicative errors:

\begin{align}
y_i & = \alpha\cdot e^{\beta\cdot x_i + \epsilon_i} + \theta \\
\text{with } \epsilon_i & \sim N(0, \sigma^2). \nonumber
\end{align}

\noindent The parameters \(\alpha\) and \(\theta\) are adjusted based on \(\beta\) and \(\sigma^2\) to guarantee the range and domain constraints are met.
The model generated \(N = 50\) points \((x_i, y_i), i = 1,...,N\) where \(x\) and \(y\) have an increasing exponential relationship.
The heuristic data generation procedure is described below:

\textit{Algorithm 2.1.1: Paremeter Estimation}

Input Parameters: domain \(x\in[0,20]\), range \(y\in[10,100]\), midpoint \(x_{mid}\).

Output: estimated model parameters \(\hat\alpha, \hat\beta, \hat\theta\)

\begin{enumerate}
\def\labelenumi{\arabic{enumi}.}
\item
  Determine the \(y=-x\) line scaled to fit the assigned domain and range.
\item
  Map the values \(x_{mid} - 0.1\) and \(x_{mid} + 0.1\) to the \(y=-x\) line for two additional points.
\item
  From the set points \((x_k, y_k)\) for \(k = 1,2,3,4\), obtain the coefficients from the linear model \(\ln(y_k) = b_0 +b_1x_k\) to obtain starting values - \(\alpha_0 = e^{b_0}, \beta_0 = b_1, \theta_0 = 0.5\cdot \min(y)\)
\item
  Using the \texttt{nls()} function from the \texttt{stats} package in Rstudio and the starting parameter values - \(\alpha_0, \beta_0, \theta_0\) - fit the nonlinear model, \(y_k = \alpha\cdot e^{\beta\cdot x_k}+\theta\) to obtain estimated parameter values - \(\hat\alpha, \hat\beta, \hat\theta.\)
\end{enumerate}

\noindent\textit{Algorithm 2.1.2: Exponential Simulation}

Input Paremeters: sample size \(N = 50\), estimated parameters \(\hat\alpha\), \(\hat\beta\), and \(\hat\theta\), \(\sigma\) standard deviation from the exponential curve.

Output Parameters: \(N\) points, in the form of vectors \(\mathbf{x}\) and \(\mathbf{y}\).

\begin{enumerate}
\def\labelenumi{\arabic{enumi}.}
\item
  Generate \(\tilde x_j, j = 1,..., N\cdot \frac{3}{4}\) as a sequence of evenly spaced points in \([0,20]\). This ensures the full domain of \(x\) is used, fulfilling the constraints of spanning the same domain and range for each parameter combination.
\item
  Obtain \(\tilde x_i, i = 1,...N\) by sampling \(N = 50\) values from the set of \(\tilde x_j\) values. This gaurantees some variability and potential clustring in the exponential growth curve disrupting the perception due to continuity of points.
\item
  Obtain the final \(x_i\) values by jittering \(\tilde x_i\).
\item
  Calculate \(\tilde\alpha = \frac{\hat\alpha}{e^{\sigma^2/2}}.\) This ensures that the range of simulated values for different standard devaition parameters has an equal expected value for a given rate of change due to the non-constant variance across the domain.
\item
  Generate \(y_i = \tilde\alpha\cdot e^{\hat\beta x_i + e_i}+\hat\theta\) where \(e_i\sim N(0,\sigma^2).\)
\end{enumerate}

\hypertarget{parameter-selection}{%
\section{Parameter Selection}\label{parameter-selection}}

For each level of difficulty, we simulated 1000 data sets of \((x_{ij}, y_{ij})\) points for \(i = 1,...,50\) and \(j = 1...10\).
Each generated \(x_i\) point from \textit{Algorithm 2.1.2} was replicated 10 times. Then the lack of fit statistic (LOF) was computed for each simulated data set by calculating the deviation of the data from a linear line.
Plotting the density curves of the LOF statistics for each level of difficulty choice allows us to evaluate the ability of differentiating between the difficulty levels and thus detecting the target plot.
In Figure \ref{fig:lof-density-curves}, we can see the densities of each of the three difficulty levels.
While the LOF statistic provides us a numerical value for discriminating between the difficulty levels, we cannot directly relate this to the perceptual discriminability; it serves primarily as an approximation to ensure that we are testing parameters at several distinct levels of difficulty.

\begin{figure}

{\centering \includegraphics[width=\linewidth]{_main_files/figure-latex/lof-density-curves-1} 

}

\caption{Lack of fit statistic density curves}\label{fig:lof-density-curves}
\end{figure}

Final parameter estimates are shown in Table \ref{tab:parameter-data}.

\begin{table}

\caption{\label{tab:parameter-data}Lineup data simulation final parameters}
\centering
\begin{tabular}[t]{ccccccc}
\toprule
 & $x_{mid}$ & $\hat\alpha$ & $\tilde\alpha$ & $\hat\beta$ & $\hat\theta$ & $\hat\sigma$\\
\midrule
Easy & 14.5 & 0.91 & 0.88 & 0.23 & 9.10 & 0.25\\
Medium & 13.0 & 6.86 & 6.82 & 0.13 & 3.14 & 0.12\\
Hard & 11.5 & 37.26 & 37.22 & 0.06 & -27.26 & 0.05\\
\bottomrule
\end{tabular}
\end{table}

\hypertarget{lineup-setup}{%
\section{Lineup Setup}\label{lineup-setup}}

Lineup plots were generated by mapping one simulated data set corresponding to difficulty level A to a scatter plot to be identified as the target plot while multiple simulated data sets corresponding to difficulty level B were individually mapped to scatter plots for the null plots.
For example, a target plot with simulated data following an increasing exponential curve with obvious curvature is embedded within null plots with simulated data following an increasing exponential trend that is almost linear (i.e.~Hard Null - Easy Target).
By our constraints, the target plot and null plots will span a similar domain and range.
There are a total of six (i.e.~\(3!\cdot 2!\)) lineup parameter combinations.
Two sets of each lineup parameter combination were simulated (total of 12 test data sets) and plotted on both the linear scale and the log scale (total of 24 test lineup plots).
In addition, there are three parameter combinations which generate homogeneous ``Rorschach'' lineups, where all panels are from the same distribution. Each participant evaluated one of these lineups, but for simplicity, these evaluations are not described in this paper.

\hypertarget{study-design}{%
\section{Study Design}\label{study-design}}

Each participant was shown a total of thirteen lineup plots (twelve test lineup plots and one Rorschach lineup plot). Participants were randomly assigned one of the two replicate data sets for each of the six unique lineup parameter combinations. For each assigned test data set, the participant was shown the lineup plot corresponding to both the linear scale and the log scale. For the additional Rorschach lineup plot, participants were randomly assigned one data set shown on either the linear or the log scale. The order of the thirteen lineup plots shown was randomized for each participant.

Participants above the age of majority were recruited from Reddit's Visualization and Sample Size communities.
Since participants recruited on Reddit were not compensated for their time, most participants have an interest in data visualization research.
Previous literature suggests that prior mathematical knowledge or experience with exponential data is not associated with the outcome of graphical experiments (S. VanderPlas \& Hofmann, 2016).
Participants completed the experiment using a Shiny applet (\url{https://shiny.srvanderplas.com/log-study/}).

Participants were shown a series of lineup plots and asked to identify the plot that was most different from the others.
On each plot, participants were asked to justify their choice and provide their level of confidence in their choice.
The goal of this experimental task is to test an individuals ability to perceptually differentiate exponentially increasing trends with differing levels of curvature on both the linear and log scale.

\hypertarget{results}{%
\section{Results}\label{results}}

Participant recruitment through Reddit occurred over the course of two weeks during which 58 individuals completed 518 unique test lineup evaluations. Previous studies have found that results do not differ on lineup-related tasks between Reddit and e.g.~Amazon Mechanical Turk (Susan VanderPlas \& Hofmann, 2017).
Participants who completed fewer than 6 lineup evaluations were removed from the study (17 participants, 41 evaluations).
The final data set included a total of 41 participants and 477 lineup evaluations.
Each plot was evaluated by between 18 and 28 individuals (Mean: 21.77, SD: 2.29).
In 67\% of the 477 lineup evaluations, participants correctly identified the target panel.

Target plot identification was analyzed using the Glimmix Procedure in SAS 9.4.
Each lineup plot evaluated was assigned a value based on the participant response (correct = 1, not correct = 0).
The binary response was analyzed using a generalized linear mixed model following a binomial distribution with a logit link function following a row-column blocking design to account for the variation due to participant and data set respectively. See model details and estimates in \ear{Appendix Reference}.

On both the log and linear scales, the highest accuracy occurred in lineup plots where the target model and null model had large curvature differences (Easy Null - Hard Target; Hard Null - Easy Target).
There is a decrease in accuracy on the linear scale when comparing a target plot with less curvature to null plots with more curvature (Easy Null - Medium Target; Medium Null - Hard Target).
Best, Smith, \& Stubbs (2007) found that accuracy of identifying the correct curve type was higher when nonlinear trends were presented indicating that it is hard to say something is linear (i.e.~something has less curvature), but easy to say that it is not linear; our results concur with this observation.
Overall, there are no significant differences in accuracy between curvature combinations when data is presented on a log scale indicating participants were consistent in their success of identifying the target panel on the log scale.
Figure \ref{fig:odds-ratio-plot} displays the estimated (log) odds ratio of successfully identifying the target panel on the log scale compared to the linear scale.
The choice of scale has no impact if curvature differences are large (Hard Null - Easy Target; Easy Null - Hard Target).
However, presenting data on the log scale makes us more sensitive to slight changes in curvature (Medium Null - Easy Target; Medium Null - Hard Target; Easy Null - Medium Target).
An exception occurs when identifying a plot with curvature embedded in null plots close to a linear trend (Hard Null - Medium Target), again supporting the claim that it is easy to identify a curve in a bunch of lines but much harder to identify a line in a bunch of curves (Best, Smith, \& Stubbs, 2007).

\begin{figure}

{\centering \includegraphics[width=\linewidth]{_main_files/figure-latex/odds-ratio-plot-1} 

}

\caption{Lineups log(odds) results}\label{fig:odds-ratio-plot}
\end{figure}

\hypertarget{discussion-and-conclusion}{%
\section{Discussion and Conclusion}\label{discussion-and-conclusion}}

The overall goal of this paper is to provide basic research to support the principles used to guide design decisions in scientific visualizations of exponential data.
In this study, we explore the use of linear and log scales to determine whether our ability to notice differences in exponentially increasing trends is impacted by the choice of scale.
Our results indicated that when there was a large difference in curvature between the target plot and null plots, the choice of scale had no impact and participants accurately differentiated between the two curves on both the linear and log scale.
However, displaying exponentially increasing data on a log scale improved the accuracy of differentiating between models with slight curvature differences.
An exception occurred when identifying a plot with curvature embedded in surrounding plots closely relating to a linear trend, indicating that it is easy to identify a curve in a group of lines but much harder to identify a line in a group of curves.
The use of visual inference to identify these guidelines suggests that there are \emph{perceptual} advantages to log scales when differences are subtle.
What remains to be seen is whether there are cognitive disadvantages to log scales: do log scales make it harder to make use of graphical information?

\hypertarget{youdrawit}{%
\chapter{Prediction with you draw it}\label{youdrawit}}

\hypertarget{estimation}{%
\chapter{Numerical Translation and Estimation}\label{estimation}}

\hypertarget{conclusion}{%
\chapter{Conclusion}\label{conclusion}}

\appendix

\hypertarget{the-first-appendix}{%
\chapter{The First Appendix}\label{the-first-appendix}}

\hypertarget{the-second-appendix}{%
\chapter{The Second Appendix}\label{the-second-appendix}}

\backmatter

\hypertarget{references}{%
\chapter*{References}\label{references}}
\addcontentsline{toc}{chapter}{References}

\noindent

\setlength{\parindent}{-0.20in}
\setlength{\leftskip}{0.20in}
\setlength{\parskip}{8pt}

\hypertarget{refs}{}
\begin{CSLReferences}{1}{0}
\leavevmode\hypertarget{ref-bavel_using_2020}{}%
Bavel, J. J. V., Baicker, K., Boggio, P. S., Capraro, V., Cichocka, A., Cikara, M., \ldots{} Willer, R. (2020). Using social and behavioural science to support {COVID}-19 pandemic response. \emph{Nature Human Behaviour}, \emph{4}(5), 460--471. http://doi.org/\href{https://doi.org/10.1038/s41562-020-0884-z}{10.1038/s41562-020-0884-z}

\leavevmode\hypertarget{ref-best_perception_2007}{}%
Best, L. A., Smith, L. D., \& Stubbs, D. A. (2007). Perception of {Linear} and {Nonlinear} {Trends}: {Using} {Slope} and {Curvature} {Information} to {Make} {Trend} {Discriminations}. \emph{Perceptual and Motor Skills}, \emph{104}(3), 707--721. http://doi.org/\href{https://doi.org/10.2466/pms.104.3.707-721}{10.2466/pms.104.3.707-721}

\leavevmode\hypertarget{ref-lisa_charlotte_2020}{}%
Charlotte, L. (2020, July). You've informed the public with visualizations about the coronavirus. Thank you. \emph{Chartable}. Retrieved from \url{https://blog.datawrapper.de/datawrapper-effect-corona/}

\leavevmode\hypertarget{ref-dehaeneLogLinearDistinct2008}{}%
Dehaene, S., Izard, V., Spelke, E., \& Pica, P. (2008). Log or {Linear}? {Distinct} {Intuitions} of the {Number} {Scale} in {Western} and {Amazonian} {Indigene} {Cultures}. \emph{Science}, \emph{320}(5880), 1217--1220. http://doi.org/\href{https://doi.org/10.1126/science.1156540}{10.1126/science.1156540}

\leavevmode\hypertarget{ref-wade_fagen_ulmschneider_2020}{}%
Fagen-Ulmschneider, W. (2020). 91-DIVOC. \emph{An interactive visualization of the exponential spread of COVID-19}. Retrieved from \url{http://91-divoc.com/pages/covid-visualization/}

\leavevmode\hypertarget{ref-financial_times_2020}{}%
Financial Times. (2020, December). Coronavirus tracked: Has the epidemic peaked near you? \emph{Coronavirus chart: see how your country compares}. Financial Times. Retrieved from \url{https://ig.ft.com/coronavirus-chart/?areas=eur}

\leavevmode\hypertarget{ref-finney1951subjective}{}%
Finney, D. (1951). Subjective judgment in statistical analysis: An experimental study. \emph{Journal of the Royal Statistical Society: Series B (Methodological)}, \emph{13}(2), 284--297.

\leavevmode\hypertarget{ref-heckler_student_2013}{}%
Heckler, A. F., Mikula, B., \& Rosenblatt, R. (2013). Student accuracy in reading logarithmic plots: {The} problem and how to fix it. In \emph{2013 {IEEE} {Frontiers} in {Education} {Conference} ({FIE})} (pp. 1066--1071). http://doi.org/\href{https://doi.org/10.1109/FIE.2013.6684990}{10.1109/FIE.2013.6684990}

\leavevmode\hypertarget{ref-hofmann_graphical_2012}{}%
Hofmann, H., Follett, L., Majumder, M., \& Cook, D. (2012). Graphical {Tests} for {Power} {Comparison} of {Competing} {Designs}. \emph{IEEE Transactions on Visualization and Computer Graphics}, \emph{18}(12), 2441--2448. http://doi.org/\href{https://doi.org/10.1109/TVCG.2012.230}{10.1109/TVCG.2012.230}

\leavevmode\hypertarget{ref-jones_polynomial_1977}{}%
Jones, G. V. (1977). Polynomial perception of exponential growth. \emph{Perception \& Psychophysics}, \emph{21}(2), 197--198. http://doi.org/\href{https://doi.org/10.3758/BF03198726}{10.3758/BF03198726}

\leavevmode\hypertarget{ref-jones_generalized_1979}{}%
Jones, G. V. (1979). A generalized polynomial model for perception of exponential series. \emph{Perception \& Psychophysics}, \emph{25}(3), 232--234. http://doi.org/\href{https://doi.org/10.3758/BF03202992}{10.3758/BF03202992}

\leavevmode\hypertarget{ref-loyVariationsQQPlots2016}{}%
Loy, A., Follett, L., \& Hofmann, H. (2016). Variations of {Q}-{Q} {Plots}: {The} {Power} of {Our} {Eyes}! \emph{The American Statistician}, \emph{70}(2), 202--214. http://doi.org/\href{https://doi.org/10.1080/00031305.2015.1077728}{10.1080/00031305.2015.1077728}

\leavevmode\hypertarget{ref-mackinnon_feedback_1991}{}%
Mackinnon, A. J., \& Wearing, A. J. (1991). Feedback and the forecasting of exponential change. \emph{Acta Psychologica}, \emph{76}(2), 177--191. http://doi.org/\href{https://doi.org/10.1016/0001-6918(91)90045-2}{10.1016/0001-6918(91)90045-2}

\leavevmode\hypertarget{ref-majumder_validation_2013}{}%
Majumder, M., Hofmann, H., \& Cook, D. (2013). Validation of {Visual} {Statistical} {Inference}, {Applied} to {Linear} {Models}. \emph{Journal of the American Statistical Association}, \emph{108}(503), 942--956. http://doi.org/\href{https://doi.org/10.1080/01621459.2013.808157}{10.1080/01621459.2013.808157}

\leavevmode\hypertarget{ref-menge_logarithmic_2018}{}%
Menge, D. N. L., MacPherson, A. C., Bytnerowicz, T. A., Quebbeman, A. W., Schwartz, N. B., Taylor, B. N., \& Wolf, A. A. (2018). Logarithmic scales in ecological data presentation may cause misinterpretation. \emph{Nature Ecology \& Evolution}, \emph{2}(9), 1393--1402. http://doi.org/\href{https://doi.org/10.1038/s41559-018-0610-7}{10.1038/s41559-018-0610-7}

\leavevmode\hypertarget{ref-mosteller_eye_1981}{}%
Mosteller, F., Siegel, A. F., Trapido, E., \& Youtz, C. (1981). Eye {Fitting} {Straight} {Lines}. \emph{The American Statistician}, \emph{35}(3), 150--152. http://doi.org/\href{https://doi.org/10.1080/00031305.1981.10479335}{10.1080/00031305.1981.10479335}

\leavevmode\hypertarget{ref-xkcd}{}%
Munroe, R. (2013, January). Log scale. \emph{xkcd}. Retrieved from \url{https://xkcd.com/1162/}

\leavevmode\hypertarget{ref-romano_scale_2020}{}%
Romano, A., Sotis, C., Dominioni, G., \& Guidi, S. (2020). \emph{The {Scale} of {COVID}-19 {Graphs} {Affects} {Understanding}, {Attitudes}, and {Policy} {Preferences}} (\{SSRN\} \{Scholarly\} \{Paper\} No. ID 3588511). Rochester, NY: Social Science Research Network.

\leavevmode\hypertarget{ref-rost_2020}{}%
Rost, L. C. (2020, July). You've informed the public with visualizations about the coronavirus. Thank you. \emph{Chartable}. Datawrapper. Retrieved from \url{https://blog.datawrapper.de/datawrapper-effect-corona/}

\leavevmode\hypertarget{ref-siegler_numerical_2017}{}%
Siegler, R. S., \& Braithwaite, D. W. (2017). Numerical {Development}. \emph{Annual Review of Psychology}, \emph{68}(1), 187--213. http://doi.org/\href{https://doi.org/10.1146/annurev-psych-010416-044101}{10.1146/annurev-psych-010416-044101}

\leavevmode\hypertarget{ref-timmers_inverse_1977}{}%
Timmers, H., \& Wagenaar, W. A. (1977). Inverse statistics and misperception of exponential growth. \emph{Perception \& Psychophysics}, \emph{21}(6), 558--562. http://doi.org/\href{https://doi.org/10.3758/BF03198737}{10.3758/BF03198737}

\leavevmode\hypertarget{ref-unwin_why_2020}{}%
Unwin, A. (2020). Why is {Data} {Visualization} {Important}? {What} is {Important} in {Data} {Visualization}? \emph{Harvard Data Science Review}. http://doi.org/\href{https://doi.org/10.1162/99608f92.8ae4d525}{10.1162/99608f92.8ae4d525}

\leavevmode\hypertarget{ref-vanderplas2020testing}{}%
Vanderplas, S., Cook, D., \& Hofmann, H. (2020). Testing statistical charts: What makes a good graph? \emph{Annual Review of Statistics and Its Application}, \emph{7}, 61--88.

\leavevmode\hypertarget{ref-vanderplasSpatialReasoningData2016}{}%
VanderPlas, S., \& Hofmann, H. (2016). Spatial {Reasoning} and {Data} {Displays}. \emph{IEEE Transactions on Visualization \& Computer Graphics}, \emph{22}(1), 459--468. http://doi.org/\href{https://doi.org/10.1109/TVCG.2015.2469125}{10.1109/TVCG.2015.2469125}

\leavevmode\hypertarget{ref-vanderplas_clusters_2017}{}%
VanderPlas, Susan, \& Hofmann, H. (2017). Clusters {Beat} {Trend}!? {Testing} {Feature} {Hierarchy} in {Statistical} {Graphics}. \emph{Journal of Computational and Graphical Statistics}, \emph{26}(2), 231--242. http://doi.org/\href{https://doi.org/10.1080/10618600.2016.1209116}{10.1080/10618600.2016.1209116}

\leavevmode\hypertarget{ref-vanderplas_statistical_nodate}{}%
Vanderplas, S., Röttger, C., Cook, D., \& Hofmann, H. (2020). Statistical significance calculations for scenarios in visual inference. \emph{Stat}. http://doi.org/\href{https://doi.org/doi.org/10.1002/sta4.337}{doi.org/10.1002/sta4.337}

\leavevmode\hypertarget{ref-varshney_why_2013}{}%
Varshney, L. R., \& Sun, J. Z. (2013). Why do we perceive logarithmically? \emph{Significance}, \emph{10}(1), 28--31. http://doi.org/\href{https://doi.org/10.1111/j.1740-9713.2013.00636.x}{10.1111/j.1740-9713.2013.00636.x}

\leavevmode\hypertarget{ref-wagenaar_misperception_1975}{}%
Wagenaar, William A., \& Sagaria, S. D. (1975). Misperception of exponential growth. \emph{Perception \& Psychophysics}, \emph{18}(6), 416--422. http://doi.org/\href{https://doi.org/10.3758/BF03204114}{10.3758/BF03204114}

\leavevmode\hypertarget{ref-wagenaar_extrapolation_1978}{}%
Wagenaar, W. A., \& Timmers, H. (1978). Extrapolation of exponential time series is not enhanced by having more data points. \emph{Perception \& Psychophysics}, \emph{24}(2), 182--184. http://doi.org/\href{https://doi.org/10.3758/BF03199548}{10.3758/BF03199548}

\leavevmode\hypertarget{ref-wickham2010graphical}{}%
Wickham, H., Cook, D., Hofmann, H., \& Buja, A. (2010). Graphical inference for infovis. \emph{IEEE Transactions on Visualization and Computer Graphics}, \emph{16}(6), 973--979.

\end{CSLReferences}


%% backmatter is needed at the end of the main body of your thesis to
%% set up page numbering correctly for the remainder of the thesis
\backmatter

%% Start the correct formatting for the appendices
% \appendix
%% Input each appendix here
% \input{./appendix_a}

%% Bibliography goes here (You better have one)
%% BibTeX is your friend

% \bibliographystyle{alpha}  % or use  abbrv to abbreviate first names and use numerical indices
\bibliographystyle{abbrv}  % or use  abbrv to abbreviate first names and use numerical indices
%% Add your BibTex file here (don't include the .bib)
\bibliography{./references}



%% Index go here (if you have one)
\end{document}
